\documentclass{article}
\usepackage{graphicx} % Required for inserting images
\usepackage{amsmath}  % Required for align environment
\usepackage{amssymb}
\usepackage{float}    % Control figure placement
\usepackage{tikz}     % Drawing and diagrams
\usepackage{hyperref}

\title{Assignment B3: State estimation}
\author{Jestin, Cormaccar, Cornal and Balaji Baskaran}
\date{6th Feburary 2025}

\begin{document}

\maketitle

\section*{State Space modeling}

\subsection*{State Equation}
% i think we have to change the dynamics, since in theory we are doing in a 2-D plane the dynamics will also change right, so based on the A matrix in therory we have to change the system dynamics as well
The state equation represents the evolution of the system's state \( \mathbf{x}_t \) over time. Considering the system to be linear, the general form is given by kinematics law:
\[
\mathbf{x}_{t+1} = \mathbf{x}_t + T_s \mathbf{v}_{t}
\]
\[ \mathbf{v}_{t+1} = \mathbf{v}_{t} + \mathbf{w}_t\]

For this system:
\[
\mathbf{x}_{t+1} = 
\begin{bmatrix}
1 & T_s \\
0 & 1
\end{bmatrix} \begin{bmatrix}
   \mathbf{x}_t \\ \mathbf{v}_{t} 
\end{bmatrix}
 +
\begin{bmatrix}
0 \\
1
\end{bmatrix}
\mathbf{w}_{t}
\]

where:
\begin{itemize}
    \item \( \begin{bmatrix} \mathbf{x}_t \\ \mathbf{v}_{t} \end{bmatrix} \): State vector, where \( \mathbf{x}_t \) is the position and \( \mathbf{v}_{t} \) is the velocity.
    \item \( T_s \): Sampling time.
    \item \( \mathbf{w}_{t} \): Process noise affecting the velocity.
\end{itemize}

\subsection*{Measurement Equation}
The measurement equation represents the relationship between the observed variable \( \mathbf{z}_{t} \) and the system's state \( \mathbf{x}_t \). The general form is:
\[
\mathbf{z}_{t} = H \mathbf{x}_t + \mathbf{u}_{t}
\]

For this system:
\[
\mathbf{z}_{t} =
\begin{bmatrix}
1 & 0
\end{bmatrix}
\mathbf{x}_t + \mathbf{u}_{t}
\]

where:
\begin{itemize}
    \item \( H = \begin{bmatrix} 1 & 0 \end{bmatrix} \): Measurement matrix.
    \item \( \mathbf{u}_{t} \): Measurement noise.
\end{itemize}

\section*{Estimation using Extended Kalman Filter (EKF)}

This model captures the system's dynamics and measurements, suitable for use in state estimation techniques like the Extended Kalman Filter (EKF). This filter uses the following equation to predict and update the system: 
\subsection*{Prediction Step (Time Update)}
\begin{align*}
\hat{x}_{t|t-1} &= f(\hat{x}_{t-1|t-1}, 0) \\
P_{t|t-1} &= A P_{t-1|t-1} A^T + Q
\end{align*}
where: 
\begin{itemize}
    \item \( \hat{x}_{t|t-1} \) is Predicted state estimate (before measurement update)
    \item \( P_{t|t-1} \) is Predicted error covariance matrix
    \item \( A \) is State transition matrix (Jacobian of \( f(x) \))
    \item \( Q \) is Process noise covariance
\end{itemize}
\subsection*{Update Step (Measurement Update)}
\begin{align*}
K_t &= P_{t|t-1} C_t^T (C_t P_{t|t-1} C_t^T + R)^{-1} \\
\hat{x}_{t|t} &= \hat{x}_{t|t-1} + K_t \left( y_t - h(\hat{x}_{t|t-1}) \right) \\
P_{t|t} &= P_{t|t-1} - K_t C_t P_{t|t-1}
\end{align*}
where:
\begin{itemize}
    \item \( K_t \) is Kalman gain
    \item \( C_t \) is Jacobian of the measurement function \( h(x) \)
    \item \( R \) is Measurement noise covariance
    \item \( \hat{x}_{t|t} \) is Updated state estimate after incorporating measurement \( y_t \)
    \item \( P_{t|t} \) is Updated error covariance
\end{itemize}
The state model has been extended for the system represents an object moving in 2D space with:
\begin{itemize}
    \item Position \( (x,y) \).
    \item Velocity \( (v_x,v_y) \).
\end{itemize}
The state equations are given as
\[\mathbf{x}_{t+1} = \mathbf{x}_t + T_s \mathbf{v}_{x}\]
\[\mathbf{y}_{t+1} = \mathbf{y}_t + T_s \mathbf{v}_{y}\]
\[v_{\mathbf{x}_{t+1}} = \mathbf{v}_x + \mathbf{w}_{x}\]
\[v_{\mathbf{y}_{t+1}} = \mathbf{v}_y + \mathbf{w}_{y}\]
where:
\begin{itemize}
    \item \(T_s\) is sampling time.
    \item \(\mathbf{w}_{x} \text{ and } \mathbf{w}_{y}\) are process noise affecting velocity.
\end{itemize}
This state vector \(\begin{bmatrix}
    \mathbf{x}_t + T_s \mathbf{v}_{x} & \mathbf{y}_t + T_s \mathbf{v}_{y}
& \mathbf{v}_x + \mathbf{w}_{x}
& \mathbf{v}_y + \mathbf{w}_{y}
\end{bmatrix}^T\) effectively captures both position and velocity, which are necessary for predicting future states in a dynamic system. The state transition matrix \(A\) is give as
\[A = \begin{bmatrix}
    1 & 0 & T_s & 0 \\ 
    0 & 1 & 0 & T_s \\
    0 & 0 & 1 & 0 \\
    0 & 0 & 0 & 1 \\ \end{bmatrix}\]
In measurement model, since we do not measure the velocity directly, the function is only depended on the position \((x,y)\). It is given as
\[
\mathbf{z}_{t} = h( \mathbf{x}_t,\mathbf{u}_{t})
\]
where \(\mathbf{z}_{t} = \begin{bmatrix}
    d_1 & d_2 & d_3
\end{bmatrix}^T + \begin{bmatrix}
    \mathbf{u}_1 & \mathbf{u}_2 & \mathbf{u}_3
\end{bmatrix}^T\) and \(d_i\) is the Euclidean distance between the vehicle's position to the \(i^{th}\) beacon. As this is a non-linear function we use Jacobian matrix \((\frac{\partial h}{\partial \mathbf{x}})\) to linearize it for the EKF update.

The estimation of the state of a system deals with uncertainties and arises from:
\begin{itemize}
    \item Process noise: The uncertainty in the system states and is given by the process noise covariance matrix \((Q)\).
    \item Measurement noise: The uncertainty in the measurement is states and is given by the measurement noise covariance matrix \((R)\).
\end{itemize}
The extended process noise matrix \(Q 
_{bar}\)is used for the full position-velocity state:
% also I think we can do exactly as how its given in the notes, like giving the equation of  Q_bar and A. Then we can say that since the system is linear, that the Jacobian becomes the A matrix and Q_bar is givrn by the formula Q_bar = GQG^T
\[Q_{bar} = \begin{bmatrix}
    0 & 0 & 0 & 0 \\ 0 & 0 & 0 & 0 \\
0 & 0 & \mathbf{w}_{x} & 0 \\ 0 & 0 & 0 & \mathbf{w}_{y} \\ \end{bmatrix}\]
The noise is added only in the velocity block and not in the position block because it does not affect the position. 

The matrix \(R\) determines the noise of the sensor's readings and as there are 3 beacons, \(R\) can be represented as \(3 \times 3\) diagonal matrix, which is:
% Check the R matrix once more, it the noise variance of the measurement and i have made it as a identity matrix
\[R = \begin{bmatrix}
     \mathbf{u}_1 & 0 & 0 \\ 0 & \mathbf{u}_2 & 0 \\
0 & 0 & \mathbf{u}_3  \end{bmatrix}\]
The filter is calculated using python running for 50 iterations which computes measurement prediction and updates the state estimate using the EKF update equations and the next state is predicted using system dynamics. 

\section*{Track Geometry}
The track consists of two arcs of a circle and two linear segments, representing a transition between one motion regime and another.

\subsection*{Track Sections}

\paragraph{First Circular Motion (Radius $r_A = 48$)}
The object moves in a counterclockwise direction along a circular path with a radius of 48 units. The interval of angles covers:
\begin{equation}
\theta = 0 \text{ to } \theta_1.
\end{equation}
The transition occurs at:
\begin{equation}
\theta_1 = 15^\circ,
\end{equation}
where the motion changes from circular to straight-line motion.

\paragraph{First Straight-Line Motion}
After the first circular segment, the object travels in a straight line at constant velocity:
\begin{equation}
v = 10 \text{ units/sec}.
\end{equation}
The endpoint of the linear segment is defined by a transition time preset.

\paragraph{Second Circular Motion (Radius $r_B = 22$)}
The object now traverses a smaller arc in a clockwise fashion with:
\begin{equation}
r_B = 22.
\end{equation}
The object follows an angular displacement up to:
\begin{equation}
\theta_B = 140^\circ.
\end{equation}

\paragraph{Second Straight-Line Motion}
Similar to the first linear segment, the second linear motion proceeds at a constant speed in a straight line. This segment ends at the beginning of the final circular arc.

\paragraph{Final Circular Motion (Radius $r_A = 48$)}
The object returns to a circular motion with the original radius:
\begin{equation}
r_A = 48.
\end{equation}
The angular displacement is:
\begin{equation}
\theta_5 = 250^\circ.
\end{equation}
The track closes back to the starting point.

\subsection*{Transition Points}
The transition points between circular and linear motion are defined by the angles and distances:
\begin{equation}
(x_1, y_1), (x_2, y_2), (x_3, y_3), (x_4, y_4)
\end{equation}
mark the end/start of circular and linear segments. These values were manually measured in the code.

\subsection*{Key Observations}
The circular segments follow the equations of a circle:
\begin{equation}
    x = r \cos(\theta), \quad y = r \sin(\theta).
\end{equation}
The linear segments follow simple kinematic equations:
\begin{equation}
    x = x_0 + v \cos(\theta) \cdot t, \quad y = y_0 + v \sin(\theta) \cdot t.
\end{equation}
The transition between sections is controlled by predefined transition times.

\section*{Time Discretization}
Time discretization is essential for simulating the motion numerically. Instead of treating time as a continuous variable, it is divided into small intervals.

\subsection*{How Discretization is Implemented}
\paragraph{Total Simulation Time}
The total simulation time is set to 43 seconds.

\paragraph{Number of Time Steps}
The motion is discretized into 500 intervals:
More intervals mean higher accuracy, but also more computations.

\paragraph{Time Step}
The time step is calculated as:
\begin{equation}
dt = \frac{\text{total time}}{\text{number of time steps}}
\end{equation}

This ensures that the simulation updates at regular time intervals.

\paragraph{Time Array ($t$)}
A numpy array stores the discrete time points:
This divides the total time into evenly spaced points for simulation.

\subsection*{Importance of time Discretization}
Larger time step ($dt$) can also speed up the computation, which means that the simulation will become more efficient. But this does not improve accuracy; it would lose significant details of the motion with a larger $dt$, causing poor tracking of fast-moving objects and possible instability of the results.

However, one needs to balance the efficiency of computing and the infidelity-induced errors, and such a decision would be application-dependent and require definition of the desired level of precision in tracking the motion with the corresponding choice of $dt$.

\end{document}
